% !TEX TS-program = xelatex
% !TEX encoding = UTF-8 Unicode
% !Mode:: "TeX:UTF-8"

\documentclass{resume}
\usepackage{zh_CN-Adobefonts_external} % Simplified Chinese Support using external fonts (./fonts/zh_CN-Adobe/)
% \usepackage{NotoSansSC_external}
% \usepackage{NotoSerifCJKsc_external}
% \usepackage{zh_CN-Adobefonts_internal} % Simplified Chinese Support using system fonts
\usepackage{linespacing_fix} % disable extra space before next section
\usepackage{cite}

\begin{document}
\pagenumbering{gobble} % suppress displaying page number

\name{罗小宝}

\basicInfo{
    \email{1944303766@qq.com} \textperiodcentered\
    \phone{18296651587}
}

\section{\faGraduationCap\  教育背景}
\datedsubsection{\textbf{西安电子科技大学}, 西安, 陕西}{2020 -- 至今}
\textit{在读硕士研究生}\ 计算机技术, 预计 2022 年 6 月毕业
\datedsubsection{\textbf{东华理工大学}, 南昌, 江西}{2016 -- 2020}
\textit{学士}\ 信息与计算科学

\section{\faUsers\ 实习/项目经历}
\datedsubsection{\textbf{广州广哈通信有限公司}, 广州}{2021年3月 -- 至今}
\role{实验室项目} {音频质量优化项目}
\begin{onehalfspacing}
    在该项目中我主要负责vad(语音端点检测)模块。主要工作为:
    \begin{itemize}
        \item 将webrtc中的vad模块抽离出来,并编译到公司平台上。
        \item 使用pytorch训练深度学习算法,最终在数据集上达到了85\%的准确率。
        \item 将深度学习算法转化成C++代码,并将其移植到公司平台上。
    \end{itemize}
\end{onehalfspacing}


\datedsubsection{\textbf{C++web服务器开发}}{2021年6月 -- 至今}
\role{C++, Linux}{个人项目}
\begin{onehalfspacing}
    使用C++11完成的web服务器.
    \begin{itemize}
        \item 利用IO复用技术Epoll与线程池实现多线程的Reactor高并发模型
        \item 利用状态机解析HTTP请求报文,实现处理静态资源的请求
        \item 实现了对于HTTP请求中的GET方法和POST方法的解析
        \item 基于小根堆实现的定时器,关闭超时的非活动连接
    \end{itemize}
\end{onehalfspacing}

\datedsubsection{\textbf{json解析器}}{2020 年6月 -- 2020年12月}
\role{C}{个人项目}
\begin{onehalfspacing}
    使用C写的json解析器,能够解析标准的json格式并且字符串化。
    \begin{itemize}
        \item 使用递归下降解析器编写了符合ECMA-404标准的JSON解析器以及生成器
        \item 该json解析器支持 UTF-8 JSON 文本和以double存储JSON number类型
    \end{itemize}
\end{onehalfspacing}

% Reference Test
%\datedsubsection{\textbf{Paper Title\cite{zaharia2012resilient}}}{May. 2015}
%An xxx optimized for xxx\cite{verma2015large}
%\begin{itemize}
%  \item main contribution
%\end{itemize}

\section{\faCogs\ IT 技能}
% increase linespacing [parsep=0.5ex]
\begin{itemize}[parsep=0.5ex]
    \item 熟悉python,了解C++/C,熟悉基本的数据结构和算法,有良好的编程风格
    \item 掌握基本的linux命令
\end{itemize}

\section{\faHeartO\ 获奖情况}
\datedline{研究生学业二等奖学金}{2021}
\datedline{研究生学业一等奖学金}{2020}

\section{\faInfo\ 其他}
% increase linespacing [parsep=0.5ex]
\begin{itemize}[parsep=0.5ex]
    \item 技术博客: https://lxb1226.github.io/
    \item GitHub: https://github.com/lxb1226
    \item 语言: 英语 - 熟练(英语六级)
\end{itemize}

%% Reference
%\newpage
%\bibliographystyle{IEEETran}
%\bibliography{mycite}
\end{document}
